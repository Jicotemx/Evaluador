\problem{A}{¿Cuánto es \(6 \times 7\)?}{42}
\problem{B}{Valor de \(\pi\) con 4 cifras decimales}{3.1416}
\problem{C}{¿Qué año es actualmente?}{2024}
\problem{D}{Calcula \(\int_0^1 3x^2 dx\)}{1}
\problema{3}{\( \begin{pmatrix} a_1 & a_2 & a_3 & a_4 & a_5 & a_6 \\ b_1 & b_2 & b_3 & b_4 & b_5 & b_6 \\ c_1 & c_2 & c_3 & c_4 & c_5 & c_6 \\ d_1 & d_2 & d_3 & d_4 & d_5 & d_6 \\ e_1 & e_2 & e_3 & e_4 & e_5 & e_6 \end{pmatrix}  \) }{:)}
\problem{Problema 1}{Calcula \(\int_0^1 x^2  dx\)}{0.33333333333}
\problem{Problema 2}{Resuelve \( \frac{d}{dx} \ln x = ? \)}{1/x}
\problem{1}{ Luis compró 8 cuadernos a 25 pesos cada uno y 7 bolígrafos. En total se gastó 298 pesos, ¿cuánto costó cada bolígrafo?}{14} 
\problem{2}{ Se han comprado gomas de borrar por un total de \(60\) pesos. Si se hubieran comprado tres gomas más, el comerciante habría hecho un descuento de \(1\) peso en cada una, y el precio total habría sido el mismo. ¿Cuántas gomas se compraron?}{12} 

